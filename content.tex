% In this file you should put the actual content of the blueprint.
% It will be used both by the web and the print version.
% It should *not* include the \begin{document}
%
% If you want to split the blueprint content into several files then
% the current file can be a simple sequence of \input. Otherwise It
% can start with a \section or \chapter for instance.

\chapter{homological algebra conventions}
\section{My proposal}
My preference would be calling any differential $\mathbb{Z}$-graded object by chain complex.
Chain complex can be indexed by either homological or cohomological indices,
with the convention that homological indices appear on the lower right corner
and cohomological indices appear on the upper right corner.
Instead of above and below, always say left and right with the convention that differentials
point to the right.
\section{To avoid confusion}
always say if an index is homological or cohoimological if it's not absolutely clear.
\subsection{just a test of subsection}

\chapter{test some references}
The glorious \cite{condensed} and \cite{analytic}!
\section{Videos}
We only have \cite{ihesvid}. Rectify it!

\chapter{definition}

See if the definition works.

\begin{definition}
\label{affine::def::with_animated_rings}
A \emph{pre-analytic animated ring} $ A $ is a pair $ (A ^{\huflag}, \D(A)) $ where $ A ^{\huflag} $ is a condensed animated ring and
$ \D (A) $ a full subcategory of $ \D (A ^{\huflag}) $ satisfying the following conditions:

\begin{enumerate}
\item The inclusion functor $ \D (A)\to \D (A ^{\huflag}) $ has a left adjoint, which we denote by $ -\dten _{A ^{\huflag}} A $,
and a right adjoint, which we denote by $ \rhoms _{A ^{\huflag}}(A, -) $;
\item If $ M,N $ are some objects of $ \D (A ^{\huflag}) $ such that $ N $ is in $\D (A) $, then $ \rhoms _{A ^{\huflag}} (M, N) $ lies in $ \D (A) $;
\item The functor $ -\dten _{A ^{\huflag}} A $, composed with the inclusion, takes $ \D ^{\leq 0}(A ^{\huflag}) $ to $ \D ^{\leq 0}(A ^{\huflag}) $;
\end{enumerate}
A pre-analytic animated ring $ (A ^{\huflag}, \D (A)) $ is called an \emph{analytic animated ring} if
the object $ A ^{\huflag} $ lies in $ \D (A) $.

A morphism of (pre-)analytic animated rings $ A\to B $ is a morphism $ f: A ^{\huflag}\to B ^{\huflag} $ of condensed animated rings such that when
$ M\in \D (B ^{\huflag}) $ lies in $ \D (B) $, the restriction of scalars $ M \in D (A ^{\huflag}) $ lies in $ \D (A) $.
\end{definition}

\section{test the dependency graph}
This is a remark but let's pretent that it's a lemma.
\begin{lemma}
\label{affine::def::compare_with_videos}
\uses{affine::def::with_animated_rings}
Instead of requiring the inclusion functor to have both left and right adjoints,
Clausen and Scholze require it to commute with all limits and colimits.
The two are equivalent in view of the reflection principle,
which is proved in \cite{adamek_reflections_1989} for ordinary categories and in \cite{ragimov_infty-categorical_2022} for $ \infty $-categories.

Indeed, when the subcategory $ \D (A) $ is closed under arbitrary limits and colimits,
the reflection principle implies that the inclusion functor $ \D (A)\to \D (A ^{\huflag}) $
has a left adjoint and that the subcategory
$ \D (A) $ is presentable.
Then the adjoint functor theorem between presentable $ \infty $-categories implies the existence of the right adjoint.
Conversely, if the inclusion functor has both left and right adjoints, then the subcategory is presentable as
a localization of a presentable $ \infty $-category. Then it has all limits and colimits,
which are taken to limits and colimits in $ \D (A) $ since the inclusion functor
is both a left and a right adjoint.

The definition of Clausen and Scholze is certainly more natural.
However I decide to bake the adjoints into the definition for the peace of mind
even though I failed to avoid invoking the reflection principle (and the adjoint functor theorem) in the sequel.
\end{lemma}

