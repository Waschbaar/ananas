\section{Condensed Sets}

\subsection{Profinite sets}

Warning: profinite set is redefined here. Classical profinite sets will be called heavy profinite sets.

Condensed sets are designed to capture convergence.
Profinite sets are the building blocks. They serve as blueprints for different convergence procedures.
Loosely a profinite set is a formal $ \mathbb{N}\opcat $-indexed limit of finite sets.
The formal limit is made precise by the free cocompletion.
\begin{definition}
The category of profinite sets, denoted by $ \profi $,
is defined to be the strictly full subcategory of $ \psh (\mathsf{Fin} \opcat)\opcat $ spanned by
the $ \mathbb{N}\opcat $-indexed colimits of objects in the Yoneda image,
where $ \mathsf{Fin} $ is the category of finite sets.

A profinite set is called finite if it is isomorphic to an object in the Yoneda image.
\end{definition}

\begin{proposition}
The category $ \profi $ is closed under countable limits and finite coproducts in $ \psh (\mathsf{Fin} \opcat)\opcat $.
In particular, $ \profi $ has countable limits and finite coproducts.
\end{proposition}

\begin{proof}

\end{proof}

A profinite set is to be thought of as an $ \mathbb{N}\opcat $-indexed limit diagram of finite sets.
If $ \mathcal{C} $ is a category with countable limits, then a fucntor $ \mathsf{Fin}\to \mathcal{C} $
can be extended to a functor $ \profi\to \mathcal{C} $ by taking the limit of the diagram in $ \mathcal{C} $.
The precise formulation for $ \mathcal{C} = \mathsf{Set} $ is:

\begin{lemma}
There is a unique faithful functor $ \profi \to \mathsf{Set} $ that commutes with countable limits
and sends a finite profinite set $ S $ to the set $ S $.
\label{profi::forgetful-to-set}
\end{lemma}

\begin{proof}

\end{proof}

\begin{definition}
The functor $ \profi \to \mathsf{Set} $ constructed in the Lemma \ref{profi::forgetful-to-set} is called the forgetful functor
and denoted by $ \mathsf{Set}(-) $.
\end{definition}

\begin{proposition}
Let $ f: X\to Y $ be a morphism in $ \profi $. The followings are equivalent:
\begin{enumerate}
\item $ \mathsf{Set}(f): \mathsf{Set}(X)\to \mathsf{Set}(Y) $ is surjective;
\item there is a profinite set $ X _{1} $ and two morphisms $ d _{0}, d _{1}: X _{1}\to X $ such that $ Y $ is the coequalizer of $ d _{0}, d _{1} $;
\item $ Y $ is the coequalizer of the two projection morphisms $ X \times _{Y} X \to X $.
\end{enumerate}
\label{profi::surj::conditions}
\end{proposition}

\begin{definition}
A morphism $ f:X\to Y $ in $ \profi $ is called surjective if the equivalent conditions in Proposition \ref{profi::surj::conditions} hold.
\end{definition}

\begin{proposition}
Let $ \{X _{i}\} $ be an $ \mathbb{N} \opcat$-index diagram of profinite sets where all transition maps $ X _{j}\to X _{i},j\geq i $ are surjective.
Then the structural maps $ \lim _{i} X _{i} \to X _{i} $ are surjective.
\end{proposition}

\begin{definition}
A morphism $ f: X\to Y $ of profinite sets is called a closed immersion if the diagonal $ X\to X\times _{Y} X $ is an isomorphism.
If $ f:X\to Y $ is a closed immersion, then $ X $ is said to be a closed sub-(profinite set) by $ f $,
or just a closed sub-(profinite set) if the morphism $ f $ is clear from context.
\end{definition}


