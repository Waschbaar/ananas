\subsection{Nöbeling's Theorem}

\begin{lemma}
\label{condensed:profinite:inj-in-prod-f2}
If $ S $ is a profinite set, then the continuous map
$ S\to \prod _{f\in \mathrm{Cont}(S, \mathbb{F}_{2})} \mathbb{F}_{2} $
determined by the maps $ f\in \mathrm{Cont}(S, \mathbb{F}_{2}) $ is injective.
\end{lemma}

\begin{proof}
Suppose that $ x, y\in S $ are distinct points.
Since $ S $ is Hausdorff, there is an open neighbourhood $ U $ of $ x $ that does not contain $ y $.
But $ S $ is also totally disconnected, so $ U $ contains a clopen neighbourhood $ V $ of $ x $ that does not contain $ y $.
The map $ g: S\to \mathbb{F}_{2} $ sending $ V $ to $ 0 $ and $ S\backslash V $ to $ 1 $ is a continuous
map that separates $ x $ and $ y $.
Thus the image of $ x, y $ in $ \prod _{f\in \mathrm{Cont}(S, \mathbb{F}_{2})} \mathbb{F}_{2} $ has different
values at coordinate $ g $.
\end{proof}

\begin{lemma}
\label{condensed:profinite:extend-loc-const-from-closed}
Suppose that $ S $ is a profinite set, $ T\subset S $ a closed subset and $ M $ a discrete topological space.
Then the map $ \mathrm{Cont}(S, M)\to \mathrm{Cont}(T, M) $
given by precomposition with the inclusion map is surjective. 
\end{lemma}

\begin{proof}
Fix a continuous map $ f: T\to M $.
Since $ T $ is compact and $ M $ is discrete, the image $ f (T)\subset M $ is finite.
Let $ T _{1}, \ldots, T _{n} $ be a partition of $ T $ such that $ f $ is constant on each of $ T _{i} $.
Each of $ T _{i} $ is closed in $ T $ and thus closed in $ S $.
Applying Lemma \ref{preliminary:topology:separate-compact-and-compact-in-profinite} $ n $-times
gives pairwise disjoint clopen subsets $ W _{1},\ldots, W _{n} $ of $ S $ such that $ T _{i}\subset W _{i} $ for each $ i $.
Define $ \tilde{f} $ as follows: if $ x\in W _{i} $ then $ \tilde{f}(x) = f (y) $ for some $ y\in T _{i}  $;
if $ x $ does not belong to any $ W _{i} $ then $ \tilde{f}(x) = f (y) $ for some $ y \in T _{1} $.
As every $ W _{i} $ is clopen, $ \tilde{f} $ is a continuous map from $ S $ to $ M $.
By construction $ \tilde{f} $ restricts to $ f $ on $ T $.
\end{proof}

\begin{theorem}
\label{profinite:noebeling}
If $ S $ is a profinite set, then the abelian group $ \mathrm{Cont}(S, \mathbb{Z}) $ is free.
\end{theorem}

\begin{proof}
In the proof the sets $ \mathrm{Cont}(-, \mathbb{Z}) $ will be equipped with the ring structures of pointwise ring operations.
Ordinals refer to the von Neumann ordinals. In particular the elements of an ordinal are (exactly instead of up to bijection) the smaller ordinals.

\textbf{Step 1: Construct the candidate for a basis.}
Suppose that $ \lambda $ is an ordinal and $ T $ is a closed subspace of the profinite set
$ \prod _{i\in \lambda} \{0, 1\} $.
For all ordinal $ \mu < \lambda $ denote by $ e _{\lambda, T, \mu} $ the continuous map $ T\to \mathbb{Z} $
the composition of the inclusion $ T\to \prod _{i\in \lambda} \{0, 1\} $,
the projection $ \prod _{i\in \lambda} \{0, 1\} \to \{0, 1\} $ at coordinate $ \mu $
and the inclusion $ \{0, 1\}\to \mathbb{Z} $.
%The map $ e _{\lambda, T, \mu} $ is an idempotent in the ring $ \mathrm{Cont}(T, \mathbb{Z}) $.

Let $ \mathcal{S}(\lambda) $ denote the set of finite sequences of ordinals of the form
$ \lambda > \mu _{1} > \mu _{2} > \cdots > \mu _{r} $. The case $ r = 0 $ is also allowed.
The set $ \mathcal{S}(\lambda) $ is ordered lexicographically. More precisely,
a sequence $ (\mu _{1}, \ldots, \mu _{r}) $ is defined to be greater than another sequence $ (\mu' _{1}, \ldots, \mu'_{r'}) $
if and only if either there is a number $ 1\leq i\leq \min \{r, r'\}$ such that $ \mu _{j} = \mu'_{j} $ for all $ j < i $
and $ \mu _{i} > \mu'_{i} $, or $ r>r' $ and $ \mu _{j} = \mu'_{j} $ for all $ 1\leq j\leq r' $.
Observe that
for $ \mu \leq \lambda $, $ \mathcal{S}(\mu)\subset \mathcal{S}(\lambda) $.
Moreover, if $ s\in \mathcal{S}(\lambda) \backslash \mathcal{S}(\mu) $ and
$ s'\in \mathcal{S}(\mu) $, then $ s > s' $.
Thus if $ s\in \mathcal{S}(\mu) $, then the sets $ \{s'\in \mathcal{S}(\lambda), s' < s\} $
and $ \{s'\in \mathcal{S}(\mu), s'<s\} $ are equal.

For every sequence $ (\mu _{1}, \ldots, \mu _{r})\in \mathcal{S}(\lambda) $, define
$ E _{\lambda, T, (\mu _{1}, \ldots, \mu _{r})} $ to be the product
$ e _{\lambda, T, \mu _{1}}\cdots e _{\lambda, T, \mu _{r}} $ in $ \mathrm{Cont}(T, \mathbb{Z}) $.
If $ r = 0 $ then $ E _{\lambda, T, ()} $ is defined to be $ 1 $.
Let $ \mathcal{S}'(\lambda, T) $ be the subset of $ \mathcal{S}(\lambda) $ consisting of
sequences $ (\mu _{1}, \ldots, \mu _{r}) $ that satisfies the following condition:
the element $ E _{\lambda, T, (\mu _{1}, \ldots, \mu _{r})} $ does not lie in the $ \mathbb{Z} $-linear span of the set
$ \{E _{\lambda, T, s}\}_{s\in \mathcal{S}(\lambda), s < (\mu _{1}, \ldots, \mu _{r})} $.
(By convention the $ \mathbb{Z} $-linear span of the empty set is $ \{0\} $.)
Define the set $ \mathscr{E}_{\lambda, T} $ to be $ \{E _{\lambda, T, s} \mid s\in \mathcal{S}'(\lambda, T)\} $.
By construction the set $ \mathscr{E}_{\lambda, T} $ is in bijection with $ \mathcal{S}'(\lambda, T) $.

Let $ \mu \leq \lambda $ be an ordinal and $ T _{\mu} $ be the image of $ T $ under the projection
$ \prod _{\nu < \lambda} \{0, 1\} \to \prod _{\nu < \mu} \{0, 1\} $.
Composition with the surjection $ T\to T _{\mu} $ is an
injective ring homomorphism $ \mathrm{Cont}(T _{\mu}, \mathbb{Z})\to \mathrm{Cont}(T, \mathbb{Z}) $.
If $ \nu\leq \mu $, then the image of $ e _{\mu, T _{\mu}, \nu} $ in $ \mathrm{Cont}(T, \mathbb{Z}) $
is $ e _{\lambda, T _{\mu}, \nu} $, which is equal to $ e _{\lambda, T, \nu} $ since the composition of
$ T\to \prod _{\lambda} \{0, 1\} $ and the projection to coordinate $ \nu $ factors through
the composition of $ T _{\mu} \to \prod _{\mu} \{0, 1\} $ and the projection to coordinate $ \nu $.
Therefore for all $ s\in \mathcal{S}(\mu) $, the image of $ E _{\mu, T _{\mu}, s} $ in $ \mathrm{Cont}(T, \mathbb{Z}) $
is $ E _{\lambda, T, s} $.
Moreover, for any $ s\in \mathcal{S}(\mu) $, the equality $ \{s'\in \mathcal{S}(\lambda), s'<s\} = \{s'\in \mathcal{S}(\mu), s'<s\} $
together with the injectivity of $ \mathrm{Cont}(T _{\mu}, \mathbb{Z})\to \mathrm{Cont}(T, \mathbb{Z}) $
implies that $ s\in \mathcal{S}'(\mu, T _{\mu}) $ if and only if $ s\in \mathcal{S}'(\lambda, T) $.
In particular, the image of $ \mathscr{E}_{\mu, T _{\mu}} $ in $ \mathrm{Cont}(T, \mathbb{Z}) $
is a subset of $ \mathscr{E}_{\lambda, T} $.

\textbf{Step 2: Transfinite induction.}
Let $ \mathscr{P}(\lambda, T) $ be the statement
\begin{center}
$ \mathscr{E}_{\lambda, T} $ is a basis of $ \mathrm{Cont}(T, \mathbb{Z}) $.
\end{center}
The statements $ \mathscr{P}(\lambda, T) $ are proved by the following transfinite induction on $ \lambda $.

\textbf{Step 2.1: Base case.}
If $ \lambda = 0 $ then $ T $ is either empty or one point.
If $ T $ is empty then $ \mathrm{Cont}(T, \mathbb{Z}) = 0 $.
The set $ \mathscr{E}_{0, T} $ is empty, a basis for $ 0 $.
If $ T $ is a point then $ \mathscr{E}_{0, T} $ is $ \{1\} $, constituting a basis in
$ \mathrm{Cont}(T, \mathbb{Z}) \simeq \mathbb{Z} $.

\textbf{Step 2.2: Limit case.}
Suppose that $ \lambda $ is a limiting ordinal.
The abelian group $ \mathrm{Cont}(T, \mathbb{Z}) $ is the directed union of the abelian groups $ \mathrm{Cont}(T _{\mu}, \mathbb{Z}) $ for $ \mu < \lambda $,
and each of the strcutural maps $ \mathrm{Cont}(T _{\mu}, \mathbb{Z})\to \mathrm{Cont}(T, \mathbb{Z}) $
sends $ \mathscr{E}_{\mu, T _{\mu}} $ to a subset of $ \mathscr{E}_{\lambda, T} $.
The induction hypothesis implies that for each $ \mu $, the set $ \mathscr{E}_{\mu, T _{\mu}} $ is a basis of $ \mathrm{Cont}(T _{\mu}, \mathbb{Z}) $.
Thus $ \mathscr{E}_{\lambda, T} $ is a basis of $ \mathrm{Cont}(T, \mathbb{Z}) $.

\textbf{Step 2.3: Successor case.}
Suppose that $ \lambda = \rho + 1 $ is a successor ordinal.
The map $ T\to \prod _{\lambda} \{0, 1\}\simeq \prod _{\rho}\{0, 1\} \times \{0, 1\} $
factors through $ T _{\rho} \times \{0, 1\} $, giving a continuous injection $ T\to T _{\rho}\times \{0, 1\} $.
Define $ T ^{i} $ to be set $ T\cap T _{\rho}\times \{i\} $ for $ i = 0, 1 $,
and $ T ^{i}_{\rho} $ the respective images in $ T _{\rho} $.
Observe that the maps $ T ^{i}\to T ^{i}_{\rho} $ are homeomorphisms and thus $ T \simeq T ^{0}_{\rho} \coprod T ^{1}_{\rho} $.

Let $ U $ be the intersection $ T ^{0}_{\rho}\cap T ^{1}_{\rho} $.
The situation can be summarized in the following commutative diagram
\begin{center}
% https://tikzcd.yichuanshen.de/#N4Igdg9gJgpgziAXAbVABwnAlgFyxMJZARgBoAGAXVJADcBDAGwFcYkQAVAPWHIF8AOgIDGENACdoAAm7BifEH1LpMufIRQAmCtTpNW7DouUgM2PASIBmHTQYs2iEEInQA+sCEBbZgB4hjPReAEZQ9HxSQrykxEIKSirm6tYxuvYGTi6SUB7efkLiABYQEVHkMXHGiWqWWql2+o6cuQJFJVWmqhYaJPV6DoY8-C1tgiJi2TI88iPF8SZmNT3lxGmN7ACqirowUADm8ESgAGaSXkjlIDgQSACsNIHBMIwACl3JTuJYe4U4IA0DTICbBeGAARw6pwg50QZCuN0QABYaIUYPQoOwcAB3CCo9EIBIgKEw7TwpA2EB4jFObG4tFQAkmYlIOHXJCknD0LCMdjFCAAa0hZyQyLJiApnO5vIgAqF0LuNDZSJR9MxOKpjJOwsQlyVcPSTSEIPBcphADZFQj7iBGFgwE04BBbRiVeikGBmIxGIquTynHzBTQ4IUsMc-rDCczEBaxdbJX7KTLAyAnmBqZdg6HwwBaeSUPhAA
\begin{tikzcd}
                                                           & T^{0}\coprod T^{1} \arrow[d, "\simeq"'] \arrow[r, "\simeq"] & T \arrow[d, two heads] \arrow[r, hook] & {\prod_{\mu<\lambda} \{0,1\}} \arrow[d, two heads] \\
U \arrow[r, hook, shift right] \arrow[r, hook, shift left] & T^{0}_{\rho}\coprod T^{1}_{\rho} \arrow[r, two heads]       & T_{\rho} \arrow[r, hook]               & {\prod_{\mu<\rho} \{0,1\}}                        
\end{tikzcd}.
\end{center}
%
%If $ U = \emptyset $,
%then the map $ T\to T _{\rho} $ is an isomorphism, and
%$ \mathrm{Cont}(T, \mathbb{Z})\simeq \mathrm{Cont}(T _{\rho}, \mathbb{Z}) $.
%The set $ \mathscr{E}_{\rho, T _{\rho}} $ is a basis of $ \mathrm{Cont}(T _{\rho}, \mathbb{Z}) $ by induction hypothesis.
%In this case the function $ e _{\lambda, T, \rho} $ is a function on $ T _{\rho} $, and thus a linear combination of elements
%of the form $ e _{\rho, T _{\rho}, \mu} $ for $ \mu < \rho $.
%So $ \mathscr{E}_{\lambda, T} $ does not contain products involving $ e _{\lambda, T, \rho} $.
%But then $ \mathscr{E}_{\lambda, T} $ is equal to the image of $ \mathscr{E}_{\rho, T _{\rho}} $, and is thus a basis of
%$ \mathrm{Cont}(T, \mathbb{Z}) $.
%
%Suppose that $ U $ is not empty.
The diagram
\begin{equation*}
U \rightrightarrows T  \to T _{\rho}
\end{equation*}
is a colimit diagram in $ \mathsf{Top} $.
The functor $ \mathrm{Cont}(-, \mathbb{Z}) $ takes colimit in $ \mathsf{Top} $
to limits in $ \mathsf{Ab} $, so the sequence
\begin{equation*}
0\to \mathrm{Cont}(T _{\rho}, \mathbb{Z}) \to \mathrm{Cont}(T, \mathbb{Z})\to \mathrm{Cont}(U, \mathbb{Z})
\end{equation*}
is exact, where the last map takes $ f:T\to \mathbb{Z} $ to $ \gamma _{1}(f) - \gamma _{0}(f) $, where $ \gamma _{i} $
is the precomposition with the map $ U \to T ^{i}_{\rho}\to T ^{0}_{\rho}\coprod T ^{1}_{\rho} \simeq T $.
The last map is also surjective by Lemma \ref{condensed:profinite:extend-loc-const-from-closed}.
Then the sequence
\begin{equation*}
0\to \mathrm{Cont}(T _{\rho}, \mathbb{Z}) \to \mathrm{Cont}(T, \mathbb{Z})\to \mathrm{Cont}(U, \mathbb{Z})\to 0
\end{equation*}
is exact.

The set $ \mathscr{E}_{\rho, T _{\rho}} $ is a basis of $ \mathrm{Cont}(T _{\rho}, \mathbb{Z}) $ by induction hypothesis.
The elements of $ \mathscr{E}_{\lambda, T} $ who are not in the image of $ \mathscr{E}_{\rho, T _{\rho}} $
are precisely those products whose first factor is $ e _{\lambda, T, \rho} $.
Consider the element $ E _{\lambda, T, (\rho, \mu _{2}, \ldots, \mu _{r})} \in \mathrm{Cont}(T, \mathbb{Z})$,
where $ (\rho, \mu _{2}, \ldots, \mu _{r})\in \mathcal{S}(\lambda) $.
Its image in $ \mathrm{Cont}(U, \mathbb{Z}) $ is
\begin{equation*}
\gamma _{1}(e _{\lambda, T, \rho})\gamma _{1}(e _{\lambda, T, \mu _{2}})\cdots \gamma _{1}(e _{\lambda, T, \mu _{r}})
- \gamma _{0}(e _{\lambda, T, \rho})\gamma _{0}(e _{\lambda, T, \mu _{2}})\cdots \gamma _{0}(e _{\lambda, T, \mu _{r}}).
\end{equation*}
But $ e _{\lambda, T, \rho} $ is $ 0 $ on $ T ^{0} $ and $ 1 $ on $ T ^{1} $,
so $ \gamma _{0}(e _{\lambda, T, \rho}) = 0 $ and $ \gamma _{1}(e _{\lambda, T, \rho}) = 1 $.
Then the image of $ E _{\lambda, T, (\rho, \mu _{2}, \ldots, \mu _{r})} $ in $ \mathrm{Cont}(U, \mathbb{Z}) $ is
\begin{equation*}
\gamma _{1}(e _{\lambda, T, \mu _{2}})\cdots \gamma _{1}(e _{\lambda, T, \mu _{r}}).
\end{equation*}
For every $ \mu< \rho $, the map 
\begin{equation*}
U\to T ^{1}_{\rho}\to T ^{0}_{\rho}\coprod T ^{1}_{\rho}\to T \to \prod _{\lambda} \{0, 1\}\xrightarrow{\text{coordinate }\mu} \{0, 1\}
\end{equation*}
and the map
\begin{equation*}
U\to T _{\rho} \to \prod _{\rho} \{0, 1\}\xrightarrow{\text{coordinate }\mu} \{0, 1\}
\end{equation*}
are the same,
and consequently $ \gamma _{1}(e _{\lambda, T, \mu}) = e _{\rho, U, \mu} $.
In particular, the image of $ E _{\lambda, T, (\rho, \mu _{2}, \ldots, \mu _{r})} $
in $ \mathrm{Cont}(U, \mathbb{Z}) $ is $ E _{\rho, U, (\mu _{2}, \ldots, \mu _{r})} $.

Suppose that for some $ (\rho, \mu _{2}, \ldots, \mu _{r}) $, the element 
$ E _{\lambda, T, (\rho, \mu _{2}, \ldots, \mu _{r})} $ is the $ \mathbb{Z} $-linear combination
\begin{equation*}
\sum _{j=1}^{n} a _{j} E _{\lambda, T, s _{j}}
\end{equation*}
where each $ s _{j} \in \mathscr{S}(\lambda) $ is strictly less than $ (\rho, \mu _{1}, \ldots, \mu _{r}) $.
Then the linear relation persists under the map $ \mathrm{Cont}(T, \mathbb{Z})\to \mathrm{Cont}(U, \mathbb{Z}) $, while
the elements $ E _{\lambda, T, s} $ where $ s $ does not start with $ \rho $ are sent to $ 0 $.
Thus $ E _{\rho, U, (\mu _{2}, \ldots, \mu _{r})} $ is a linear combination of various $ E _{\rho, U, s'} $
for $ s'\in \mathcal{S}(\rho), s' < (\mu _{2}, \ldots, \mu _{r}) $.

Conversely, suppose that for some $ (\mu _{1}, \ldots, \mu _{r})\in \mathcal{S}(\rho) $, the element
$ E _{\rho, U, (\mu _{1}, \ldots, \mu _{r})} $ is a linear combination
\begin{equation*}
\sum _{j=1}^{n} a _{j} E _{\rho, U, s _{j}}
\end{equation*}
where each $ s _{j}\in \mathscr{S}(\rho) $ is strictly less than $ (\mu _{1}, \ldots, \mu _{r}) $.
Then the image of
\begin{equation*}
F = E _{\lambda, T, (\rho, \mu _{1}, \ldots, \mu _{r})} - \sum _{j=1}^{n} a _{j} E _{\lambda, T, (\rho, s _{j})}
\end{equation*}
(the notation $ (\rho, s _{j}) $ means $ (\rho, \nu _{1}, \ldots, \nu _{r'}) $ where $ s _{j} = (\nu _{1}, \ldots, \nu _{r'}) $)
in $ \mathrm{Cont}(U, \mathbb{Z}) $ is equal to $ 0 $.
But then $ F $ lies in the image of $ \mathrm{Cont}(T _{\rho}, \mathbb{Z})\to \mathrm{Cont}(T, \mathbb{Z}) $
and is thus a linear combination of various $ E _{\lambda, T, s} $ where $ s'\in \mathscr{S}(\lambda) $ does not start with $ \rho $.
Such an $ s' $ is strictly less than $ (\rho, \mu _{1}, \ldots, \mu _{r}) $.
Thus $ E _{\lambda, T, (\rho, \mu _{1}, \ldots, \mu _{r})} $ is a linear combination of various $ E _{\lambda, T, s} $
where $ s < (\rho, \mu _{1}, \ldots, \mu _{r}) $.

Moreover, if two elements $ E _{\lambda, T, (\rho, \mu _{1},\ldots, \mu _{r})} $ and $ E _{\lambda, T, (\rho, \mu'_{1},\ldots,\mu'_{r'})} $
with $ (\rho,\mu _{1},\ldots,\mu _{r}) > (\rho, \mu'_{1},\ldots,\mu'_{r'}) $ are sent to the same image in $ \mathrm{Cont}(U, \mathbb{Z}) $,
then $ E _{\lambda, T, (\rho, \mu _{1}, \ldots,\mu _{r})} $ is a linear combination of $ E _{\lambda, R, (\rho, \mu'_{1},\ldots,\mu'_{r'})} $
and various $ E _{\lambda, T, s} $ where $ s $ does not start with $ \rho $. Thus $ E _{\lambda, T, (\rho, \mu _{1},\ldots, \mu _{r})} $ can not be
in the set $ \mathscr{E}_{\lambda, T} $.
In conclusion, the set $ \mathscr{E}_{\lambda, T}\backslash \mathscr{E}_{\rho, T _{\rho}} $
is sent bijectively to the set $ \mathscr{E}_{\rho, U} $.

Invoking the induction hypothesis $ \mathscr{P}(\rho, U) $,
the image of $ \mathscr{E}_{\lambda, T}\backslash \mathscr{E}_{\rho, T _{\rho}} $
is a basis of $ \mathrm{Cont}(U, \mathbb{Z}) $.
Then the short exact sequence splits, and $ \mathscr{E}_{\lambda, T} $ is a basis of $ \mathrm{Cont}(T, \mathbb{Z}) $.
This concludes the transfinite induction.

\textbf{Step 3: Apply to $ S $.}
Choose a well-ordering for the set $ \mathrm{Cont}(S, \mathbb{F}_{2}) $,
which amounts to a bijection of $ \mathrm{Cont}(S, \mathbb{F}_{2}) $ with an ordinal $ \lambda $.
The theorem follows from the statement $ \mathscr{P}(\lambda, S) $, thanks to Lemma \ref{condensed:profinite:inj-in-prod-f2}.
\end{proof}

\begin{remark}
Asgeirsson in \cite{noebelingformal} constructs a Lean-verified formal proof of Nöbeling's theorem.
However, be warned that the proof presented here is not (yet) verified by Lean or any other computer-assisted proof checking tool.
\end{remark}

